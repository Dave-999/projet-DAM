\documentclass[a4paper,oneside,12pt]{report}

\usepackage{custom}

\newcommand{\barg}{\si{\bar}\text{g}} 

\usepackage{fancyhdr} 
\fancyhf{}
\fancyhead[R]{P3 – 2014 - gr25 }   
\fancyfoot[C]{\thepage}                     
\renewcommand\headrulewidth{0pt}
\pagestyle{fancy}

\begin{document}

\begin{titlepage}

LMECA1200 - Projet en construction mécanique 1
Dimensionnement d’une bielle
année académique 2014-2015

\end{titlepage}

\section{Réponse aux questions}

Le fonctionnement d'un moteur à explosion repose sur la transmission du mouvement alternatif du piston, mettant le vilebrequin en rotation. Cela se fait par l'intermédiaire d'une bielle. Cette pièce, répétant un cycle à raison de plusieurs milliers de fois par minutes, subit des forces conséquentes. Il est donc primordial d'en faire l'analyse afin de prévoir sa forme optimale et les efforts maximaux à devoir supporter. C'est cette tâche que nous réaliserons dans ce bref rapport. Nous partons du cas particulier du moteur d'une Audi A4 dont nous avons hérité du vilebrequin lors des séances de mesures.

\subsection{Mesures}
1. Sur base des mesures réalisées au laboratoire sur votre pièce et les pièce associée,
déterminez les grandeurs géométriques du moteur :
D;R;L;V
c
. Pour le taux de
compression,
$\tau$
, une recherche personnelle est nécessaire. Il dépend de la nature
du carburant utilisé, essence ou diesel.\\

\textit{Je m'en charge (Dave)}

\subsection{Evolution de la pression}
2. Calculez l’évolution de la pression dans le cylindre en intégrant numériquement
la relation (8). Pour la phase de combustion, utilisez un angle de démarrage de
15
o
avant le PMH et une durée de
40
o
. L’énergie apportée par la combustion dé-
pend de la nature du combustible. Pour un moteur à essence prenez
2800
kJ=kg
.
Pour un moteur diesel, prenez
1650
kJ=kg
. Comme il s’agit d’un apport de cha-
leur, la masse de référence est celle de l’air dans le cylindre.
2800
kJ=kg
corres-
pond donc à
2800
kJ
par kilogramme d’air dans le cylindre. Le gaz parcourant
le cycle est supposé diatomique avec une valeur du coefficient isentropique,
,
de 1.3 pour tenir compte de l’effet de la température sur les chaleurs massiques.

\subsection{Efforts sur la bielle}
3. Calculez ensuite les efforts sur la bielle en fonction de l’angle de vilebrequin.
Ces efforts dépendent de la vitesse de rotation. Faites le calcul pour une vitesse
normale (3000 rpm (revolutions per minute) pour un moteur à essence et 2500
rpm pour un moteur diesel) et pour une vitesse élevée (respectivement 5000 rpm
et 4000 rpm). Illustrez l’évolution des efforts sur un cycle complet du moteur.
Cherchez les efforts maximaux et minimaux qui s’exercent sur la bielle.

\subsection{Justification de la forme de la bielle}
4. Justifiez la forme en "I" du corps de la bielle

\subsection{Dimensionnement de la bielle}
5. Dimensionnez la section de la bielle (efforts de flambage). A nouveau, une re-
cherche personnelle sera nécessaire pour faire le lien entre les forces évaluées
et la forme de la bielle. Comparez vos calculs aux mesures faites sur les pièces
réelles.

\subsection{Analyses complémentaires}
Optionnel

\end{document}
